\documentclass[
    author={Jacob Daniel Halsey},
    supervisor={Prof. Awais Rashid},
    degree={BSc},
    title={Building a Testbed for Evaluating Privacy Enhancing Technologies  (PETs)},
    subtitle={},
    type={software development},
    year={2021}
]{dissertation}

\usepackage[backend=biber,bibstyle=ieee,citestyle=numeric]{biblatex}
\bibliography{dissertation}

\begin{document}

\maketitle
\frontmatter
\makedecl
\tableofcontents

\chapter*{Executive Summary}

The goal of this project is to produce a simple and lightweight testbed platform for evaluating privacy enhancing
technologies.
It should provide support for testing varied architectures and network topologies, such as client-server and
peer-to-peer applications.
It should also support simulating applications for different types of platforms including mobile phone apps.

\vspace{1cm}

Summary of my work:

\begin{itemize}
\item I have developed a flexible command line tool called \emph{kvm-compose} for Linux using the Rust
      language and \emph{libvirt} library for building and destroying virtual testing environments.
\item In the process I have made some contributions to the \emph{libvirt-rust} language
      bindings open source library.
\item I have then implemented some example projects using the testbed tool.
\end{itemize}

\chapter*{Supporting Technologies}

\begin{itemize}
    \item \emph{Linux KVM} (Kernel-based Virtual Machine) - \url{https://www.linux-kvm.org/}
    \item \emph{Open vSwitch} Virtual multilayer switch - \url{https://www.openvswitch.org/}
    \item \emph{libvirt} Virtualization API - \url{https://libvirt.org/}
    \item \emph{Rust} Language, Compiler, Toolchain, etc. - \url{https://www.rust-lang.org/}
    \item \emph{libvirt-rust} Rust bindings to the libvirt - \url{https://gitlab.com/libvirt/libvirt-rust}
    \item \emph{clap} Rust command Line Argument Parser - \url{https://github.com/clap-rs/clap}
    \item \emph{serde} Rust Serialization framework - \url{https://github.com/serde-rs/serde}
    \item \emph{serde-yaml} YAML backend for serde - \url{https://github.com/dtolnay/serde-yaml}
    \item \emph{serde-plain} Plain text backend for serde - \url{https://github.com/mitsuhiko/serde-plain}
    \item \emph{thiserror} Rust error derive macro - \url{https://github.com/dtolnay/thiserror}
    \item \emph{anyhow} Rust error handling framework - \url{https://github.com/dtolnay/anyhow}
    \item \emph{simple\_logger} Rust logging implementation - \url{https://github.com/borntyping/rust-simple_logger}
    \item \emph{xml-rs} XML library for Rust - \url{https://github.com/netvl/xml-rs}
    \item \emph{validator} Rust struct validation - \url{https://github.com/Keats/validator}
    \item \emph{directories} User data directories library - \url{https://github.com/dirs-dev/directories-rs}
    \item \emph{reqwest} Rust HTTP Client - \url{https://github.com/seanmonstar/reqwest}
    \item \emph{indicatif} Rust command line progress indicator - \url{https://github.com/mitsuhiko/indicatif}
    \item \emph{tempfile} Rust temporary file library - \url{https://github.com/Stebalien/tempfile}
    \item \emph{casual} Rust user input parser - \url{https://github.com/rossmacarthur/casual}
    \item \emph{derive-new} Rust new constructor macro - \url{https://github.com/nrc/derive-new}
    \item \emph{enum-iterator} Rust macro for iterating enums - \url{https://github.com/stephaneyfx/enum-iterator}
    \item \emph{rust-embed} Embeds files into Rust binaries - \url{https://github.com/pyros2097/rust-embed}
\end{itemize}

\chapter*{Acknowledgements}

I would like to thank my supervisor Professor Awais Rashid and co-supervisor Joe Gardiner for their
project proposal and support and guidance in completing it.

\mainmatter

\chapter{Contextual Background}
\label{chap:context}

The UK Research and Innovation (UKRI) is a non-departmental public body of the United Kingdom Government
sponsored by the Department for Business, Energy and Industrial Strategy~\cite{ukri_who_we_are}.
In October 2020 the UKRI announced the creation of the National Research Centre on Privacy, Harm Reduction
and Adversarial Influence Online (REPHRAIN)~\cite{ukri_new_centre}.
The centre is made up of researchers in computer science, international relations, law, psychology, management,
design, digital humanities, public policy, political Science, criminology, and sociology from five British
universities including the University of Bristol.

\vspace{1cm}
\noindent
This chapter should describe the project context, and motivate each of
the proposed aims and objectives.  Ideally, it is written at a fairly 
high-level, and easily understood by a reader who is technically 
competent but not an expert in the topic itself.

In short, the goal is to answer three questions for the reader.  First, 
what is the project topic, or problem being investigated?  Second, why 
is the topic important, or rather why should the reader care about it?  
For example, why there is a need for this project (e.g., lack of similar 
software or deficiency in existing software), who will benefit from the 
project and in what way (e.g., end-users, or software developers) what 
work does the project build on and why is the selected approach either
important and/or interesting (e.g., fills a gap in literature, applies
results from another field to a new problem).  Finally, what are the 
central challenges involved and why are they significant? 
 
The chapter should conclude with a concise bullet point list that 
summarises the aims and objectives.  For example:

\begin{quote}
\noindent
The high-level objective of this project is to reduce the performance 
gap between hardware and software implementations of modular arithmetic.  
More specifically, the concrete aims are:

\begin{enumerate}
\item Research and survey literature on public-key cryptography and
      identify the state of the art in exponentiation algorithms.
\item Improve the state of the art algorithm so that it can be used
      in an effective and flexible way on constrained devices.
\item Implement a framework for describing exponentiation algorithms
      and populate it with suitable examples from the literature on 
      an ARM7 platform.
\item Use the framework to perform a study of algorithm performance
      in terms of time and space, and show the proposed improvements
      are worthwhile.
\end{enumerate}
\end{quote}

\chapter{Technical Background}
\label{chap:technical}

{\bf A compulsory chapter,     of roughly $10$ pages} 
\vspace{1cm} 

\noindent
This chapter is intended to describe the technical basis on which execution
of the project depends.  The goal is to provide a detailed explanation of
the specific problem at hand, and existing work that is relevant (e.g., an
existing algorithm that you use, alternative solutions proposed, supporting
technologies).  

Per the same advice in the handbook, note there is a subtly difference from
this and a full-blown literature review (or survey).  The latter might try
to capture and organise (e.g., categorise somehow) {\em all} related work,
potentially offering meta-analysis, whereas here the goal is simple to
ensure the dissertation is self-contained.  Put another way, after reading 
this chapter a non-expert reader should have obtained enough background to 
understand what {\em you} have done (by reading subsequent sections), then 
accurately assess your work.  You might view an additional goal as giving 
the reader confidence that you are able to absorb, understand and clearly 
communicate highly technical material.

\chapter{Project Execution}
\label{chap:execution}

{\bf A topic-specific chapter, of roughly $15$ pages} 
\vspace{1cm} 

\noindent
This chapter is intended to describe what you did: the goal is to explain
the main activity or activities, of any type, which constituted your work 
during the project.  The content is highly topic-specific, but for many 
projects it will make sense to split the chapter into two sections: one 
will discuss the design of something (e.g., some hardware or software, or 
an algorithm, or experiment), including any rationale or decisions made, 
and the other will discuss how this design was realised via some form of 
implementation.  

This is, of course, far from ideal for {\em many} project topics.  Some
situations which clearly require a different approach include:

\begin{itemize}
\item In a project where asymptotic analysis of some algorithm is the goal,
      there is no real ``design and implementation'' in a traditional sense
      even though the activity of analysis is clearly within the remit of
      this chapter.
\item In a project where analysis of some results is as major, or a more
      major goal than the implementation that produced them, it might be
      sensible to merge this chapter with the next one: the main activity 
      is such that discussion of the results cannot be viewed separately.
\end{itemize}

\noindent
Note that it is common to include evidence of ``best practice'' project 
management (e.g., use of version control, choice of programming language 
and so on).  Rather than simply a rote list, make sure any such content 
is useful and/or informative in some way: for example, if there was a 
decision to be made then explain the trade-offs and implications 
involved.

\section{Example Section}

This is an example section; 
the following content is auto-generated dummy text.
\lipsum

\subsection{Example Sub-section}

\begin{figure}[t]
\centering
foo
\caption{This is an example figure.}
\label{fig}
\end{figure}

\begin{table}[t]
\centering
\begin{tabular}{|cc|c|}
\hline
foo      & bar      & baz      \\
\hline
$0     $ & $0     $ & $0     $ \\
$1     $ & $1     $ & $1     $ \\
$\vdots$ & $\vdots$ & $\vdots$ \\
$9     $ & $9     $ & $9     $ \\
\hline
\end{tabular}
\caption{This is an example table.}
\label{tab}
\end{table}

\begin{algorithm}[t]
\For{$i=0$ {\bf upto} $n$}{
  $t_i \leftarrow 0$\;
}
\caption{This is an example algorithm.}
\label{alg}
\end{algorithm}

\begin{lstlisting}[float={t},caption={This is an example listing.},label={lst},language=C]
for( i = 0; i < n; i++ ) {
  t[ i ] = 0;
}
\end{lstlisting}

This is an example sub-section;
the following content is auto-generated dummy text.
Notice the examples in Figure~\ref{fig}, Table~\ref{tab}, Algorithm~\ref{alg}
and Listing~\ref{lst}.
\lipsum

\subsubsection{Example Sub-sub-section}

This is an example sub-sub-section;
the following content is auto-generated dummy text.
\lipsum

\paragraph{Example paragraph.}

This is an example paragraph; note the trailing full-stop in the title,
which is intended to ensure it does not run into the text.

\chapter{Critical Evaluation}
\label{chap:evaluation}

{\bf A topic-specific chapter, of roughly $15$ pages} 
\vspace{1cm} 

\noindent
This chapter is intended to evaluate what you did.  The content is highly 
topic-specific, but for many projects will have flavours of the following:

\begin{enumerate}
\item functional  testing, including analysis and explanation of failure 
      cases,
\item behavioural testing, often including analysis of any results that 
      draw some form of conclusion wrt. the aims and objectives,
      and
\item evaluation of options and decisions within the project, and/or a
      comparison with alternatives.
\end{enumerate}

\noindent
This chapter often acts to differentiate project quality: even if the work
completed is of a high technical quality, critical yet objective evaluation 
and comparison of the outcomes is crucial.  In essence, the reader wants to
learn something, so the worst examples amount to simple statements of fact 
(e.g., ``graph X shows the result is Y''); the best examples are analytical 
and exploratory (e.g., ``graph X shows the result is Y, which means Z; this 
contradicts [1], which may be because I use a different assumption'').  As 
such, both positive {\em and} negative outcomes are valid {\em if} presented 
in a suitable manner.

\chapter{Conclusion}
\label{chap:conclusion}

{\bf A compulsory chapter,     of roughly $5$ pages} 
\vspace{1cm} 

\noindent
The concluding chapter of a dissertation is often underutilised because it 
is too often left too close to the deadline: it is important to allocation
enough attention.  Ideally, the chapter will consist of three parts:

\begin{enumerate}
\item (Re)summarise the main contributions and achievements, in essence
      summing up the content.
\item Clearly state the current project status (e.g., ``X is working, Y 
      is not'') and evaluate what has been achieved with respect to the 
      initial aims and objectives (e.g., ``I completed aim X outlined 
      previously, the evidence for this is within Chapter Y'').  There 
      is no problem including aims which were not completed, but it is 
      important to evaluate and/or justify why this is the case.
\item Outline any open problems or future plans.  Rather than treat this
      only as an exercise in what you {\em could} have done given more 
      time, try to focus on any unexplored options or interesting outcomes
      (e.g., ``my experiment for X gave counter-intuitive results, this 
      could be because Y and would form an interesting area for further 
      study'' or ``users found feature Z of my software difficult to use,
      which is obvious in hindsight but not during at design stage; to 
      resolve this, I could clearly apply the technique of Smith [7]'').
\end{enumerate}

\backmatter
\printbibliography

\end{document}
